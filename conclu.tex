\section{Conclusion}
\par
The white paper document addresses various important issues and does a good job of categorizing approaches and discussing solutions. One area in which the white paper document completely overlooks is the view of the laws with respect to cryptography. Encryption and other cryptographic protocols form an extremely important facet to computer security. The document overlooks and doesn't ask questions about what cryptographic assumptions should be considered secure under the law. If we were to cryptographically secure the data, then, will we be able to send it across borders? If yes, which protocols are the ones on which we can concentrate, for effective communication in India? If there are such protocols that exist in the future (blockchain technology is one interesting primitive that seems to be a good secure alternative for keeping databases secure and usage more transparent), should they be made mandatory in law for the welfare of the citizens? Even Europe's GDPR doesn't define these notions explicitly \cite{secproto}. Their idea of anonymizing or pseudonymizing is very different from cryptographic techniques which rely on well worked out security proofs and well studied mathematical assumptions. These technologies are used widely in financial banking transactions, healthcare to general surfing over the internet. The government should look at these modern techniques and make provisions for these in the law.