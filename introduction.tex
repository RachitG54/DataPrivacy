\section{Disclaimer}
The views contained in the report are the personal views of the student and do not endorse or reflect the views of the institute.
\section{Introduction}
The vision of our country to be a digital economy and perform business on a global electronic scale necessitates the creation of laws that secure the data of individuals. A firm legal framework for data protection is the foundation on which data-driven innovation and entrepreneurship can flourish in India. Fostering such innovation and entrepreneurship is essential if India is to lead its citizens and the world into a digital future committed to empowerment, experiment, and equal access.
\par
It is believed that by 2020, the global volume of digital data we create is expected to reach 44 zettabytes \cite{digitalworld}. Enterprises around the world have realized the value of these databases and are continually investing in technology for its proper mining and use. There are algorithms being developed to find patterns within data to decide on activities which can be beneficial to society. There are also a lot of opportunities for businesses to improve and capture a larger share of the market using such approaches. An example of this was the Netflix data challenge. The original intention of the competition was to generate an algorithm that can better be suited for their business and beneficial to the users. But the anonymized Netflix data could be easily combined with other data sets such as timestamps with public information from the Internet Movie Database(IMDb) to de-anonymize the original data set and reveal personal movie choices \cite{netflix}. Such instances warn us about this fast-moving field and warrant us to be cautious as well.
\par
India presently does not have any express legislation governing data protection or privacy. The relevant laws in India dealing with data protection are the Information Technology Act, 2000 and the (Indian) Contract Act, 1872. Under section 43A of the Indian Information Technology Act,2000, if an organization is negligent in maintaining security practices or uses them for wrongful gain may be held liable to pay damages to the person so affected. These laws are not explicit enough in their statements. The specificities about the actual pay damages, the definition of personal, sensitive information or the technicalities of what is constituted as negligence are presently unclear in the law. The issue is that either the details are not explicitly stated or they overlook some key details which are ever-growing as new data-driven technologies hit the market.
\par
Technology convenience and benefits are something that even the courts have realized. The Delhi high court recently accepted a WhatsApp notification as receipt proof \cite{whatdoublereceipt}. After a long battle, the Supreme court in its verdict in Puttaswamy acknowledged the need of data privacy laws. The Supreme Court in Puttaswamy \cite{puttaswam} overruled its previous judgments of M.P. Sharma v.Satish Chandra (M.P. Sharma) \cite{mpsharma} and Kharak Singh v. State of Uttar Pradesh (Kharak Singh) \cite{kharak} and held that Article 21 of the Constitution of India is the repository of residuary personal rights and it recognized the right to privacy. But even with this precedent, without a proper law, the courts are regularly being presented with cases and issues \cite{whatuserthird}.
\par
\phantomsection \label{havasupai}
These incidents call for an urgent need for these laws. Laws must be made keeping in mind the views and beliefs of Indians and must work well with an increasingly interconnected world. In 1989, researchers from Arizona State University partnered with the Havasupai Tribe, a community with high rates of Type II Diabetes, to study links between genes and diabetes risk. The researchers were unsuccessful in finding concrete patterns but they used the same data for other studies such as schizophrenia, migration, and inbreeding. These topics are considered taboo by the tribe \cite{havasupai1, havasupai2}. Such incidents should remind us of the fact that we need to take into account a communities' beliefs when processing their private data. We need to be aware and cognizant of the comparative and international practices.
\par
Even with this urgent need, we must be careful of the laws that are created and do not make decisions in haste. In the recent past, the Indian government has not been careful with the privacy of user data. The Aadhaar project, which is one of the worlds largest identity projects, has had serious privacy concerns \cite{aadharlok}. The Aadhaar Act enables the Government to collect identity information from citizens \cite{aadhar30}, including their biometrics, issue a unique identification number or an Aadhaar Number on the basis of such biometric information \cite{aadhar3}, and thereafter provide targeted delivery of subsidies, benefits, and services to them \cite{aadhar7}. The project received a lot of criticism from social commentators and civil society activists \cite{aadharbhanu,aadhararun,aadharjean} due to the way the privacy was seen and implemented. Although there were a few suggestions \cite{aadharlist} and recommendations from Shah (The Planning Commission: Government of India, 2011), Sinha (Lok Sabha Secretariat: New Delhi, 2012) committees \cite{aadharshah} and the computer science community \cite{aadharshw}, the present security of the Aadhaar project continues to be under scrutiny with serious concerns. \cite{aadharnew}.
\par
Data protection norms for personal information collected under the Aadhaar Act are also found in the Aadhaar (Data Security) Regulations, 2016 (Aadhaar Security Regulations). The Aadhaar Security Regulations impose an obligation on the UIDAI to have a security policy which sets out the technical and organizational measures which will be adopted by it to keep information secure. But the present developments suggest that these policies still might not be enough \cite{aadharshw}, and we need to be really careful and keep in mind the implementation aspects of such a project along with the laws.