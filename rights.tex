\section{Grounds of Processing, Obligation on Entities and Individual Rights}
\subsection{Consent}
\begin{itemize}
	\item What are your views on relying on consent as a primary ground for processing personal data?
	\\
	An individual's consent should be valid only if an individual could reasonably expect to understand the nature, purpose, and consequences of the collection, use or disclosure of the personal information to which she has consented.
	\\ 
	Consent should be a primary ground for processing. User consent is a must for any use of their data. As the technology is expanding at an accelerated pace, we might not be able to foresee what processing might mean in the future. Individuals may be able to foresee an immediate harm caused by misuse of their personal information, however, it is highly unlikely that they will be able to predict future uses of their information, which takes place after combining it with other data sets. Also, there are many users that fall into the trap of consent fatigue and multiplicity of notices. Hence in general consent should be a necessary but not sufficient ground for processing. Reason must be stated clearly as to what this data is to be used for. For any alteration or additional use, new consent must be taken from the user.
	\item What should be the conditions for valid consent? Should specific requirements such as ‘unambiguous’,  ‘freely given’ etc. as in the EU GDPR be imposed? Would mandating such requirements be excessively onerous?
	\\
	Consent that is given should be clear in its articulation, revocable and auditable.
	\\
	It seems logical to assume implicit consent in cases where the data is not sensitive and require for explicit consent when processing personal data. All transactions may not warrant the same standards of consent. Therefore, there is a need to explore and accommodate standards of consent within the data protection law and align it with different types of information. These details should also be evolved and updated through the legislation at regular intervals.
	\\
	These requirements might be onerous for non-sensitive data, but for sensitive data these details are mandatory.
	\item How can consent fatigue and multiplicity of notices be avoided? Are there any legal or technology-driven solutions to this?
	\\
	A service might share a video indicating what kind of data will be used by the service, although mandating such a video in this law is not feasible. It is something that the industry should thrive towards as they move forward. Other technological solutions might include designing a parser that can parse through the more relative parts of the documents, and mentions alternate options for the legal terms.
	\\
	In terms of legal driven solutions, there can be a committee that looks into these notices and regulates these so that they are not as confusing to the end user. These notices are extremely long and the jargon used in these cases is often very complicated. The data controller who processes the data must get these notices passed by this community.
	\item Would, having very stringent conditions for obtaining valid consent be detrimental to day-to-day business activities? How can this be avoided?
	\\
	Obtaining valid consents do exist in many of the current businesses settings. There might be some businesses that will get affected by these laws, but soon these will become a standard.
\end{itemize}
\subsection{Child's Consent}
\begin{itemize}
\item What are your views regarding the protection of a child's personal data?
\\
An individual's consent should be valid only if an individual could reasonably expect to understand the nature, purpose, and consequences of the collection, use or disclosure of the personal information to which she has consented. Keeping this view in mind, distinct provisions could be carved out within the data protection law which prohibit the processing of children's personal data for potentially harmful purposes, such as profiling, marketing, and tracking.
\item Should the data protection law follow the South African approach and prohibit the processing of any personal data relating to a child, as long as she is below the age of 18, subject to narrow exceptions?
\\
No, I believe this to be an extremely strict notion for students to not use the internet. This also is against digital learning.
\item If a subjective test is used in determining whether a child is capable of providing valid consent, who would be responsible for conducting this test?
\\
The entity which collects the information. The test though should carefully be set in place by the data protection authority.
\item How can the requirement for parental consent be operationalized in practice? What are the safeguards which would be required?
\\
Parental consent needs to be secure. It can be linked to the parent's identity using KYC and OTPs. Another way in which a child's consent can be obtained is through zero-knowledge proofs which might be able to verify their age without revealing explicit information about their identity. These solutions can be combined to ensure informed consent.
\end{itemize}
\subsection{Storage Limitation and Data Quality}
\begin{itemize}
	\item What are your views on the principles of storage limitation and data quality?
	\\
	Storage limitation in my view is necessary to prevent storing mass data about an individual. Presently there are no checks in place, this can lead to enterprises or governments storing large chunks of data about individuals. With a big data revolution and newer algorithms, there is no say at what stage storing excessive data might make a person identifiable. In the case of Cambridge Analytica, we saw how this lead to influence an individuals choice of voting and decide on their sexual orientations \cite{cambana}. All data processers must be allowed to share, or collude this data with anyone or any external party for any business purpose. 
	\item On whom should the primary onus of ensuring the accuracy of data lie especially when consent is the basis of collection?
	\\
	The onus should be over to the data controller, and the data owner should have the right to access and rectify, modify its data.
	\item How long should an organization be permitted to store personal data? What happens upon completion of such time period?
	\\
	Any organization must state upfront the duration for which user data is being used/ stored, after that the data must be completely erased for the benefit and security of user. Even during use, the data should be anonymized under the constraint of informed consent from the user. In such a case the user must be aware of the exact data fields which will be removed and which part of their data will be remembered.
\end{itemize}
\subsection{Individual Participation Rights - 1}
\begin{itemize}
	\item Should there be a fee imposed on exercising the right to access and rectify one‘s personal data?
	\\
	This depends on the usage of the personal data. If the personal data is a consequence of a fundamental right and a necessary part of an individuals identity such as the details in Aadhaar, changing them should be free of cost. On the other hand, in the commercial sector, this fees can be asked and should be under regulation so that companies don't deny the users by asking for outrageous amounts. For small startups, there can be provisions in the government where the government promotes usage of services that will be suitable for the right to access and hence allow startups to participate in ensuring such facilities and thus incorporating such functions in the industry.
	\item What should be the scope of the right to rectification? Should it only extend to having inaccurate data rectified or should it include the right to move to court to get an order to rectify, block, erase or destroy inaccurate data as is the case with the UK?
	\\
	Yes, it should include clauses which court orders can remove. A person should have the right to rectify their data, this includes right to edit, replace, or erase their data. Simply correcting out of date data will not be a sufficient application of the law. To achieve this, there must be a system to recognise a request for rectification and an understanding of the application of this right. There must in accordance be procedures or policies that ensure that proper response to such requests are ensured. This should be coupled with a receipt ensuring that the appropriate changes were made or valid/justifiable explanations for refusal of such a request.
	\item Is guaranteeing a right to access the logic behind automated decisions technically feasible? How should India approach this issue given the challenges associated with it?
	\\
	I don’t think this is presently feasible. If India were to encourage moving towards such a functionality. Services that allow access to large-scale data which can be centrally (usable by the data controller), dynamically(subject to changes), and securely(data should be secure from malicious users) should be promoted in India.
\end{itemize}
\subsection{Individual Participation Rights - 2}
\begin{itemize}
	\item The EU GDPR introduces the right to restrict processing and the right to data portability. If India were to adopt these rights, what should be their scope?
	\\
	The right to data portability is extremely important and shouldn't be restricted. The same is true for the right to restrict processing, however, for both, there might be some explicit exceptions stated and a time period be provided so that the law can be upheld properly.
	\item Should there be a prohibition on evaluative decisions taken on the basis of automated decisions?
	\\
	There should be a right to object to automated decisions as is the case with the UK.
\end{itemize}

\subsection{Individual Participation Rights - 3: Right to be forgotten}
\begin{itemize}
	\item Does a right to be forgotten, add any additional protection to data subjects not already available in other individual participation rights?
	\\
	Yes, this right is an extremely important addition and it does indeed offer additional protection to subjects. It is also worth noting that this functionality has not been noted anywhere else in the white paper document, and a right such as this is essential in order to ensure privacy.
	\item Are there any alternative views on this?
	\\
	I believe it is extremely necessary to formally recognize the ways in which the data is forgotten. Whether it can still be accessed by anyone. Like in the verdict of Google Spain the data can be simply accessed by opening a VPN connection \cite{googlespain}. Also with many Wayback Machine sites available which contain the information about the internet at a time in the past. To accurately define forgotten will be important \cite{wayback}.
\end{itemize}