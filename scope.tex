\section{Scope And Exemptions}
\subsection{Territorial and Personal Scope}
\begin{itemize}
	\item What are your views on what the territorial scope and the extra-territorial application of a data protection law in India?
	\\
	For territorial scope, the primary test for applicability of law should be processing of personal information which takes place in the territory of India. This should cover Indian entities and non-Indian entities that do not have a presence in India but process the data associated with the Indian population.
	\\
	Extraterritorial application of data protection must ensure that the data associated with any Indian entity is adequately protected and must observe the enforceability of such a law. Recently, a five-judge constitution bench headed by Chief Justice Dipak Misra did not pass an order but directed WhatsApp to file an affidavit giving details of what user data was shared with third parties and other entities within four weeks \cite{whatuserthird}. In this particular case, on 25 August 2016, the users were sent a notification by WhatsApp asking them to accept the changes in terms and conditions. This allowed them to share this data with Facebook for commercial use. This draws attention to the fact that these notifications should not be left to the consent of users who fall into the trap of consent fatigue and multiplicity of notices.
	\\
	India is a very diverse nation with endless varieties of physical features and cultural patterns. We cannot expect non-territorial laws to keep in mind the unique beliefs of different cultures. Particularly, we would not want our sensitive personal data to be used to challenge beliefs, as it happened in the case of the people of Havasupai tribe(see above \ref{havasupai}) \cite{havasupai1,havasupai2}
	\item To what extent should the law be applicable outside the territory of India in cases where data of Indian residents are processed by entities who do not have any presence in India?
	\\
	There should be strict regulation of data in cases where sensitive data of Indian residents are involved. For data which does not come under the category of sensitive data, this becomes more a question of cross-border data flow or imposing laws that regulate usage of data within foreign nations. 
	\\
	In the former scenario of cross-border data flow the two approaches mentioned are the adequacy test and the comparable level of protection for personal data which will be determined by a data protection authority. This requires a strong, well-formed and updated data protection authority. The latter scenario is less feasible as it has problems with enforceability; it also makes for the entity doing business to be caught among laws from various places which can lead to contradictions and legal hassles. Hence we must be careful in the distinction, it may be reasonable to ask a foreign company to abide by a country's abuse-prevention rules (rules that prevent unauthorized use of personal data), but it is less feasible to impose on the company the duty of designating a Data Protection Officer. One such solution is to use a hybrid multi-layered approach as mentioned in \cite{layeredapp}.
\item While providing such protection, what kind of link or parameters or business activities should be considered?
\\
Covering of cases where processing wholly or partly happens in India, irrespective of the status of the entity, is an approach which does not cover the full application of the law. The data of Indian people can come under a scenario where the processing is completely done outside. An Indian user might be accessing a site or company based in the USA and can get into conflicts with them. Regulating entities which offer goods or services in India even though they may not have a presence in India (modeled on the EU GDPR) seems to be the correct view here. Regulation should be subjective so that it does not demote further business opportunities.
\item What measures should be incorporated in the law to ensure effective compliance by foreign entities inter alia when adverse orders (civil or criminal) are issued against them?
\\
In such a case a sensible approach is to adopt the penalties the EU GDPR prescribes based on global turnover \cite{eugp83}. Further, a failure to pay fines or to comply with any other sanctions imposed by the law could be linked to an order restricting market access \cite{dataindo}.
\end{itemize}
\subsection{Other issues of scope}
\begin{itemize}
	\item What are your views on the issues relating to the applicability of a data protection law in India in relation to: 
	\begin{itemize}
		\item natural/juristic person 
		\item public and private sector
		\item retrospective application
	\end{itemize} 
	of such a law?
	\\
	The laws should be applied to a natural person. For juristic individuals, a separate analysis needs to be done. This can be considered in future versions of the law. 
	\\
	Public and private sectors should have a common law. The Kharak Singh vs State of Uttar Pradesh (1962) case \cite{kharak} brought into light the same dilemma. Public institutions should only be granted exemptions under very specific instances.
	\\
	Retrospective application of the law is extremely necessary. However, there needs to be special care taken to define what measures can be taken on these instances. These measures need to be consistent enough so that their application is possible and the time taken to implement those measures is also taken into account.
	\item Should the law seek to protect data relating to juristic persons in addition to protecting personal data relating to individuals?
	\\
	The law could regulate personal data of natural persons alone for the time being. It is coherent with most data protection legislations around the world. If it was chosen to apply these, it creates issues. It may definitely provide better protection to companies for their confidential business data. It may also, however, create conflict between contractual arrangements like the confidentiality agreements and the law, similar to those seen with licensing agreements and copyright law.
	\item Should the law be applied to government/public and private entities processing data equally? If not, should there be a separate law to regulate government/public entities collecting data?
	\\
	Have a common law imposing obligations on Government and private bodies as is the case in most jurisdictions. Legitimate interests of the State can be protected through relevant exemptions and other provisions. The Right to Privacy should be a fundamental right and only under very specific instances which are listed down explicitly should the public sector be exempted.
	\item Should the law provide protection retrospectively? If yes, what should be the extent of retrospective application? Should the law apply in respect of lawful and fair processing of data collected prior to the enactment of the law?
	\\
	The law will apply to processes such as storing, sharing, etc. irrespective of when data was collected, while some requirements such as grounds of processing may be relaxed for data collected in the past.
	\\
	A method needs to be devised to deal with such past illegal collections of data. We can see an example in the Facebook case where a decision must be made on what happens to the data already disclosed to Facebook \cite{whatuserthird}.
	\item Should the law provide for a time period within which all regulated entities will have to comply with the provisions of the data protection law?
	\\
	Yes, we must give adequate time for the industry to adapt to these regulations. The law that is decided will have a lot of facets and can be complicated to comprehend. Since there might be components in the law that need to be implemented securely to the end user, the industry will need time to meet the required regulations.
\end{itemize}
\subsection{Definition of Personal Data}
\begin{itemize}
	\item For the purpose of a data protection law, should the term  ‘personal data’ or  ‘personal information’ be used?
	\\
	Adopt one term, personal data as in the EU GDPR or personal information as in Australia, Canada or South Africa.
	\item What kind of data or information qualifies as personal data? Should it include any kind of information including facts, opinions or assessments irrespective of their accuracy?
	\\
	The idea to qualify identifiable data as personal data seems to be the correct approach. But as the technology is expanding, social media is exploding and innovative ideas are coming up, it seems difficult as to what may be identifiable. There might be patterns in the random data points which might be learned. An example where this confusion can clearly be seen is the Netflix data challenge \cite{netflix}.
	The data was anonymized by the problem setters but it was realized later that the data could "identify" the people when combined with a public IMDb dataset. The most challenging question here is to adequately define what will be included in the data so that privacy is ensured. 
	\\
	It should include any kind of facts, opinions or assessments keeping in mind the accuracy of the statements. Anything that can either lead to defamation, identification or breach of a person's belief system. Defaming people on social media by stating incorrect facts, opinions and assessments can lead to a questionable reputation in the community. Modified photos that scandalize persons or businesses are clear defamation violation and are quite popular on social media. It is common for modified photos or video to go 'viral'. In such a case, the accuracy of the statements made should be taken into account.
	\\ 
	Selling of opinions and facts by companies to other organizations so that targeted ad campaigns can be made and public opinion be modeled is one of the new weapons of the digital age. The algorithm used in the Facebook data breach trawled through personal data for information on sexual orientation, race, gender – and even intelligence and childhood trauma. It also claimed that it can influence voters by targeting them with personalized ads \cite{cambana}. Another case where a communities belief system was overlooked was the case of the Havasupai tribe(See above \ref{havasupai}) \cite{havasupai1, havasupai2}. 
	\item Should anonymized or pseudonymized data be outside the purview of personal data? Should the law recommend either anonymization or pseudonymization, for instance as the EU GDPR does?
	\\
	From a cryptographic standpoint, simply removing the identity as in anonymization is not enough for disguising identities. A study has shown that it’s possible to personally identify 87 percent of the U.S. population based on just three data points: five-digit ZIP code, gender, and date-of-birth \cite{zipcodeid}. Thus, what constitutes as identifiable is difficult to answer and this places huge concerns on data. Cryptographic protocols that can securely de-anonymize datasets and allow computation of different functions on any kind of datasets are far from being efficient today(Functional Encryption \cite{bsw11} and Fully Homomorphic Encryption \cite{gen09}).
	Advances in this field will really open new possibilities and the government should keep an eye on these.
	\\
	Anonymization is not enough as seen in the scenario of Netflix users \cite{netflix} and pseudonymization is an even weaker outlook towards things. The EU GPDR places a greater emphasis on anonymization and introduces notions of pseudonymization in many scenarios \cite{anonpseudo}. Our laws should also de-segment the cases and mention specifically which approach should be taken and when. Special care needs to be taken on the techniques that will be used for these purposes and that they are implemented by experts in the field only.
\end{itemize}
\subsection{Definition of Sensitive Personal Data}
\begin{itemize}
	\item Should the law define a set of information as sensitive data? If yes, what category of data should be included in it? Eg. Financial Information / Health Information / Caste / Religion / Sexual Orientation. Should any other category be included?
	\\
	Financial Information, Health Information, Sexual Orientation should be included in sensitive personal data. Categories such as caste and religion are also sensitive and can be a cause for discrimination in our society. But we need to be careful in the formulation of the law as to what will categorize as sensitive information. A person's name can, in many cases, be related to their caste or religion, and shouldn't be included as sensitive but explicitly mentioning their caste or religion should be. Genetic data and biometric data should also be included in these categories as these form an integral part of our identity. Also with the Aadhaar Act and phone authentication functions, there is a huge importance placed on a person's biometric data. An increasing amount of sensitive biometric data is also being collected, whether through fingerprints, face patterns for authentication, DNA collection, or fitness apps tracking traits.
	\\
	Data regarding the physical location of a person is also a category which should be added. It is extremely sensitive information about a person's whereabouts. Travel apps or taxi cab apps should not make this data public or sell these to third parties. There was one such incident where a journalist was threatened by Uber executives using privately collected user data \cite{ubersmear}.
	\\
	Under financial information, the law needs to be clear as to what details should be included. The transaction histories of a person, their purchases and bill payments presently can easily be disclosed to third parties. 
	%This has lead to mobile wallet companies allowing unsecured loans for facilitating usage of their wallets \cite{payloan}.
\end{itemize}
\subsection{Definitions of Processing}
\begin{itemize}
	\item Should the definition of processing list only main operations of processing i.e. collection, use and disclosure of data, and inclusively cover all possible operations on data?
	\\
	I believe it should inclusively cover all possible operations. Exclusively mentioning might miss operations that are innovated later. Also, this gives the law a broader coverage to regulate on.
	\item Should the scope of the law include both automated and manual processing? Should the law apply to manual processing only when such data is intended to be stored in a filing system or in some similar structured format?
	\\
	If data is collected manually, only filing systems should be covered as the risk of
	profiling is lower in other cases. This is the feasible way to go about things. The option of including all personal data processed, however, it may be processed, is infeasible in terms of upholding the law as we as individuals are always manually processing data. Limiting to digital or automated records is against the retrospective application and limits the applicability of the law in many scenarios in India where digitization is not the main method of data collation.
\end{itemize}
\subsection{Definition of Data Controller and Processor}
\begin{itemize}
	\item Should the law only define 'data controller' or should it additionally define 'data processor'?
	\\
	Use the two concepts of 'data controller' and 'data processor' (an entity that receives information) to distribute primary and secondary responsibility for privacy, because in many cases data is outsourced to entities.
	\item How should responsibility among different entities involved in the processing of data be distributed?
	\\
	There should be a clear bifurcation of roles and associated expectations from various entities. The expectations might not be explicitly clarified contrary to the case of EU GDPR as the compliance costs on data processors can be high \cite{obproc}.  Concerns relating to the enforceability of contracts and enforcement capabilities in India must also be taken into account. The law can identify different entities and relate them to broad expectations and not be stringent in its application.
\end{itemize}
\subsection{Exemptions}
\begin{itemize}
	\item What are your views on including research/historical/statistical purpose as an exemption?
	\\
	The kind of research that we can conduct as an exemption needs to keep in mind the sentiments of the data owner. In the late 1980s, researchers violated the sentiments of the Havasupai tribe. They claimed to use this data for diabetes research but instead used it for unrelated topics and studies about migration, inbreeding, schizophrenia. These topics are considered taboo by the Tribe and hence were an ethical violation of the feelings of an entire community. (See above \ref{havasupai}). Also, the usage of this data for commercial purposes needs to be taken care of. Today India's Aadhaar act has led to the collection of one of the largest databases in the country. Protected sharing of this data can help immensely in research/historical/statistical purposes.
	\\
	I strongly believe that exemptions only through proper approvals should take place and we need to keep in mind the different beliefs and the consent of the people. As India is a diverse country, these beliefs are often subjective and hence the exemptions should be seen on a case to case basis.
\end{itemize}
\subsection{Cross Border Flow of Data}
\begin{itemize}
	\item Should the data protection law have specific provisions facilitating cross-border transfer of data? If yes, should the adequacy standard be the threshold test for transfer of data?
	\\
	Yes, there should be laws in place for facilitating cross-border transfer of data. Such cross-border flow of data can be seen in the case of BPOs. A global data flow can play an important role in promoting trade, research, and development in the country. Although we should be careful to not send the sensitive data across the border. The adequacy standard seems a good test for deciding the countries where the transfer of data can occur too. We can decide on countries which have good implementations of data privacy laws and trust their legal system to uphold the rights of Indians. If we create more stringent laws then enforcing it on foreign territory might not be easy.
	\item Should certain types of sensitive personal information be prohibited from being transferred outside India even if it fulfills the test for transfer?
	\\
	Yes, the sensitive personal information should be prohibited from being transferred outside the country. This includes financial information, health information, genetic build up and all the categories which come under the sensitive data umbrella. These are private data and due to the complications that arise with the enforcement of laws across boundaries. One such instance where problems can clearly be seen is the case of Microsoft against the US government. Microsoft refused to provide the details of certain emails, basing their legal argument on the fact that the data requested was stored in servers outside US \cite{usmicro}.
\end{itemize}
\subsection{Data Localization}
\begin{itemize}
	\item Should there be a data localization requirement for the storage of personal data within the jurisdiction of India? If yes, what should be the scope of the localization mandate? Should it include all personal information or only sensitive personal information?
	\\
	In my view, it must include all the data generated by users in the digital domain. Localised storage and processing of sensitive data is mandatory as currently there isn't a unified framework for data protection laws throughout the world. This also prevents foreign surveillance and easier law enforcement that will take into account the varied beliefs of the people of this country \cite{usmicro}. 
	\\
	This, however, can lead to increased local surveillance. Large-scale government surveillance should not be allowed on this data, only under specific assumptions should the government be given controlled access to the data.
	\item If the data protection law calls for localization, what would be the impact on industry and other sectors?
	\\
	One impact of data localization on industry and other sectors is that regulations hinder the new and upcoming startups. Local storage and processing means that the low-cost benefits of cloud computing and other data storage services cannot be used, hence the cost of performing efficiently and maintaining themselves now lies on the head of the small startups. This also prevents companies from around the world in investing in the data of our nation. In the future, if big data algorithms turn extremely successful, China being a closed market would mean that India would be the natural destination for investments to gather data. Here, we must lay down the law carefully so that we do not slow development in hope for stringent laws.
	\\
	Even in the light of above advantages, local storage will allow for more stern control by the regulator. All data should be stored and processed on systems located within India. We should not compromise security in hope for increased investment.
\end{itemize}